\chapter{Introduction to Radio Astronomy}

\section{Introduction}
Radio astronomy involves the study of radio waves from the depths of space. Many objects in the universe emit radio waves through naturally occurring processes. Such objects include stars, galaxies, and nebulae, as well as a wide variety of peculiar, fascinating, and often mysterious objects, such as pulsars and quasars. Radio astronomers study such objects using a variety of radio telescopes. Many of the astronomical objects that emit radio waves do not emit much or any light so that radio astronomers study what is essentially an invisible universe not seen by even the world’s largest optical telescopes. \\

The development of radio astronomy in the mid-twentieth century opened a new window on the universe. Before this development, astronomical observations were confined to the narrow range of visible wavelengths, limiting the range of astronomical phenomena that could be studied. The sky at radio wavelengths is vastly different from the sky observed in visible light. Stars, which are the primary sources of visible light, do not dominate the emission in the radio sky. \\

At radio wavelengths, we can detect thermal continuum and spectral-line emission from objects too cold to emit visible light. This enables the study of the cold interstellar medium of our Galaxy and others, as well as the cosmic microwave background, the relic radiation from the early universe. Nonthermal radiation, such as synchrotron emission, produces significant radio emission and is observed in a variety of astronomical sources like supernova remnants and quasars. For instance, two of the brightest sources in the radio sky, Cassiopeia A and Cygnus A, are synchrotron-emitting sources that are relatively faint at visible wavelengths. Thus, radio observations complement optical observations. \\

\section{Historical Development and Milestones}
The field of radio astronomy began in the early 20th century, revolutionizing our understanding of the universe by exploring it beyond visible wavelengths. The foundation for radio astronomy was laid in the 19th century through advances in electricity and magnetism, particularly with Maxwell's equations, which revealed the vast spectrum of electromagnetic radiation.

\subsection{Early Efforts and Discoveries}
In 1887, Heinrich Hertz successfully produced radio waves in a laboratory, sparking interest in detecting cosmic radio waves. Pioneering efforts in the 1890s by Thomas Edison, Arthur Kennelly, and Sir Oliver Lodge to find correlations between sunspots and radio signals failed due to the insensitivity of the equipment used. Other astronomers like Johannes Wilsing, Julius Scheiner, and Charles Nordmann also faced similar setbacks.

\subsection{Breakthroughs in the 1930s}

The critical advancement came in 1932 when Karl Jansky of Bell Labs successfully detected astronomical radio emissions while studying static interference in short-wave radio communications. Using a steerable antenna, Jansky identified a steady hiss with a directional dependence, tracing it to the plane of the Milky Way and pinpointing the galactic center as a primary source. His work, however, initially received little attention.

Intrigued by Jansky's findings, Grote Reber, an engineer, constructed a 30-foot parabolic antenna in his backyard in 1937. After initial failures at higher frequencies, Reber succeeded in mapping the galaxy's radio emissions at 160 MHz. His detailed maps identified significant sources, including the strong emission from the galactic center and secondary peaks in Cassiopeia (Cas A) and Cygnus (Cyg A). Reber's work, published in 1940 and 1944, marked the first radio wavelength observations in an astronomical journal.

\subsection{Advancements During and After World War II}

World War II brought significant technological advancements in radar, leading to critical developments in radio astronomy. In 1942, J.S. Hey of the British Army Operational Research Group linked radar jamming to sunspot activity, a hypothesis later supported by Southworth of Bell Labs, who detected radio emissions from the quiescent Sun.

Post-war, Hey and colleagues continued their radio studies, discovering in 1945 that meteor trails reflect radio waves and mapping the radio sky in greater detail than Reber. They identified Cygnus A as a discrete source and in 1948 linked solar radio bursts to sunspots and solar flares.

\subsection{The Hydrogen Line and Interferometry}

In 1944, Dutch astrophysicist Jan Oort suggested to Hendrik van de Hulst the calculation of the hydrogen emission line due to electron spin-flip transitions. Van de Hulst predicted radiation at a 21-cm wavelength, first detected by Harold Ewen and Edward Purcell at Harvard in 1951. This discovery enabled detailed hydrogen mapping in the Milky Way and remains a vital tool in radio astronomy.

In 1946, Martin Ryle and D.D. Vonberg conducted the first interferometric observations using paired radio antennas. Subsequent interferometry by various researchers provided precise positions for bright radio sources, facilitating optical identifications of objects like Cygnus A, Cassiopeia A, and the Crab Nebula. By 1959, the Third Cambridge Catalog (3C) cataloged the brightest 471 radio sources, later revised as 3CR.

\subsection{Theoretical Insights and Nobel Prizes}

The 1950s saw the understanding of synchrotron radiation, a process where relativistic electrons emit photons while spiraling around magnetic field lines, proposed by Hannes Alfvén, Nicolai Herlofsen, and Iosif Shklovsky. The 1960s brought significant discoveries, including quasars, pulsars, and the cosmic microwave background (CMB). These discoveries led to several Nobel Prizes: the 1974 prize to Anthony Hewish for the discovery of pulsars and the 1978 prize to Arno Penzias and Robert Wilson for the CMB. Further studies on pulsars and the CMB earned additional Nobel Prizes in 1993 (Joseph Taylor and Russell Hulse) and 2006 (John Mather and George Smoot).
